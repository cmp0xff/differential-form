\documentclass[a4paper]{article}

\input{./preambles/preamble}
\usetikzlibrary{external,
angles,
decorations.pathmorphing,
calc,intersections,through,backgrounds,
decorations.pathreplacing,
patterns,shadings,
arrows,
shapes.geometric}
% save compiled tikz plots; enable --shell-escape
\tikzexternalize[prefix=./tikz/]
% Sebastian did it
\usepackage[outline]{contour}
\contourlength{2pt}

\input{./preambles/unicode}
	\setmathfont{XITS Math}[range={
		"02A30-"02A3E}] % ти╝ etc.

\setdefaultlanguage{english}
\setotherlanguages{german,french,italian,greek,latin,russian}
\setmainfont{CMU Serif}
\setsansfont{CMU Sans Serif}
% \newfontfamily{\greekfont}{CMU Serif}
% \newfontfamily{\greekfontsf}{CMU Sans Serif}

\input{./preambles/math-single}
\input{./preambles/math-brac}
\input{./preambles/math-thm}
\input{./preambles/phys-chem}

\newcommand{\RomaN}[1]{%
  \textup{\uppercase\expandafter{\romannumeral#1}}%
}

\usepackage[
	bibencoding  = auto,
	backend      = biber,
	sorting      = nyt,
	sortlocale   = auto,
	hyperref     = true,
	backref      = true,
	style        = phys,
	doi          = false,
	eprint       = true,
	isbn         = true,
	url          = true,
	subentry     = true,
	articletitle = true,
	biblabel     = brackets,
	chaptertitle = true,
	pageranges   = true,
	%refsegment   = section,
]{biblatex}
\addbibresource{main.bib}

%\input{./preambles/biblatex-user}

% \setcounter{secnumdepth}{6}
% So that paragraph is also numbered

%\numberwithin{equation}{subsection}

\title{Tool-kit for the coframe formalism}
\author{Yi-Fan Wang}


\begin{document}
\maketitle

%\begin{abstract}
%Your abstract.
%\end{abstract}

\tableofcontents

%1234567890123456789012345678901234567890123456789012345678901234567890123456789
\section{Introduction}

%1234567890123456789012345678901234567890123456789012345678901234567890123456789

This note is intended to be pragmatic, collecting a set of toolkits for the 
calculation of field theories without and with gravitation. The physical 
motivation will be discussed elsewhere.

%1234567890123456789012345678901234567890123456789012345678901234567890123456789
\section{Complex scalar field}

%1234567890123456789012345678901234567890123456789012345678901234567890123456789

The action reads
\begin{align}
S &= \int -\dd\phi^*\wedge\star\dd\phi - m^2 \phi^*\wedge\star\phi
\\
&= \int -\frac{1}{2}\rbr{\dd\phi^*\wedge\star\dd\phi 
		+ \dd\phi\wedge\star\dd\phi^*}
		- \frac{1}{2} m^2 \rbr{\phi^*\wedge\star\phi
		+ \phi\wedge\star\phi^*}
\end{align}

Variation commutes with exterior derivative
\begin{align}
\sbr{\dva, \dd}\phi = 0\,.
\end{align}

In the absence of gravitation, variation also commutes with Hodge star. A 
generic variation of the kinetic terms therefore reads
\begin{align}
\begin{split}
\dva \rbr{\dd \phi^*\wedge\star\dd\phi} &= 
\dd\rbr{\dva\phi^*\wedge\star\dd\phi + \dva\phi\wedge\star\dd\phi^*}
\\
&\quad\,+
\dva\phi^*\wedge\star\dd^{\dagger}\dd\phi +
\dva\phi\wedge\star\dd^{\dagger}\dd\phi^*\,,
\end{split}
\end{align}
where the codifferential is defined in \cref{eq:codiff-def-10}.
A generic variation of the action reads
\begin{align}
\dva S &=
\int \dd\rbr{-\dva\phi^*\wedge\star\dd\phi -
\dva\phi\wedge\star\dd\phi^*}
\nonumber \\
&\quad\,-
\dva\phi^*\wedge\star\rbr{\dd^{\dagger}\dd+m^2}\phi -
\dva\phi\wedge\star\rbr{\dd^{\dagger}\dd+m^2}\phi^*\,.
\end{align}

%123456789012345678901234567890123456789
\subsection[$U(1)$ Noether current]{$\msansU(1)$ Noether current}

%123456789012345678901234567890123456789

The action is invariant
\begin{align}
\dva_{\lambda} S \equiv 0
\end{align}
under the rigid transformation
\begin{align}
\phi \to \ee^{-\ii e \lambda} \phi\,,\qquad
\phi^* \to \ee^{+\ii e \lambda} \phi^*\,.
\end{align}
When the equations of motion are satisfied, infinitesimal 
transformation leads to
\begin{align}
0 &= \int \lambda\, \dd \mfrakJ_0\,,
\nonumber \\
\mfrakJ_0 &\coloneqq \ii e\rbr{\phi^*\wedge\star\dd\phi
	- \phi\wedge\star\dd\phi^*}\,,
\end{align}
which is the Noether current, a twisted $3$-form, satisfying the continuity 
equation
\begin{align}
\dd \mfrakJ_0 = 0\,.
\end{align}

%123456789012345678901234567890123456789
\subsection[$T(D)$ Noether current]{$\msansT(D)$ Noether current}

%123456789012345678901234567890123456789
The action is invariant up to a total differential
\begin{align}
\dva_{\lambda} S &= \int - \lambda\,\dd\rbr{z \intprod \mfrakL}
\nonumber \\
&= \int \lambda\,\frac{1}{2}\,\dd\cbr{
	+ z\intprod\sbr{\dd\phi^*\wedge\star\dd\phi +
	\dd\phi\wedge\star\dd\phi^* +
	m^2\rbr{\phi^*\,\star\phi
	+\phi\,\star\phi^*}}}\,,
\label{eq:tot-dif-ckg-10}
\end{align}
under the rigid infinitesimal transformation
\begin{align}
\dva\phi = -\lambda\, \pounds_z \phi = -\lambda\, z \intprod \dd \phi\,,\qquad
\dva\phi^* = -\lambda\, \pounds_z \phi^* = -\lambda\, z \intprod \dd \phi^*\,.
\end{align}
Note the minus sign, which has to do with our mostly-positive metric convention 
$\rbr{-, +, \ldots, +}$.

When the equations of motion are satisfied, infinitesimal 
transformation leads to
\begin{align}
\dva_{\lambda} S = \int \lambda\,\dd\cbr{
	+\rbr{z\intprod\dd\phi^*}\wedge\star\dd\phi
	+\rbr{z\intprod\dd\phi}\wedge\star\dd\phi^*}\,.
\end{align}
Taking the difference with \cref{eq:tot-dif-ckg-10} and setting $z = 
\mscre_{\alpha}$ yield
\begin{align}
0 &= \int \lambda\,\dd \mfrakT_{\alpha}, \\
\begin{split}
2 \mfrakT_{\alpha} &\coloneqq
+\rbr{\mscre_{\alpha}\intprod\dd\phi^*}\wedge\star\dd\phi +
\dd\phi^*\wedge\rbr{\mscre_{\alpha}\intprod\star\dd\phi}
\\
&\quad\,+
\rbr{\mscre_{\alpha}\intprod\dd\phi}\wedge\star\dd\phi^* +
\dd\phi\wedge\rbr{\mscre_{\alpha}\intprod\star\dd\phi^*}
\\
&\quad\,-
m^2\rbr{\phi^*\,\mscre_{\alpha}\intprod\star\phi +
	\phi\,\mscre_{\alpha}\star\phi^*}\,.
\end{split}
\end{align}
The continuity equation reads
\begin{align}
\dd \mfrakT_{\alpha} = 0\,.
\end{align}
In components,
\begin{align}
\mfrakT_{\alpha} =
+\sbr{\rfun{\mscre_{\alpha}}{\phi^*} \rfun{\mscre^{\beta}}{\phi}+
	\rfun{\mscre_{\alpha}}{\phi} \rfun{\mscre^{\beta}}{\phi^*}}\,
		\epsilon_{\beta} -
\sbr{\rfun{\mscre^{\beta}}{\phi^*} \rfun{\mscre_{\beta}}{\phi^*}
	+ m^2 \phi^* \phi}\, \epsilon_{\alpha}\,.
\end{align}

%1234567890123456789012345678901234567890123456789012345678901234567890123456789
\section{Pure electromagnetic field}

%1234567890123456789012345678901234567890123456789012345678901234567890123456789

The action reads
\begin{align}
S = \int -\frac{1}{2} F \wedge \star F\,.
\end{align}
Variation of the Lagrangian reads
\begin{align}
\dva \rbr{F\wedge\star F} &=
	2\,\dva F \wedge \star F
\\ &=
	2\,\dd\rbr{\dva A \wedge \star F} + 2\,\dva A \wedge \dd {\star F}\,.
\end{align}
Note that this is not gauge invariant.

\subsection{Lorenz gauge}
The Laplace--de Rham operator, or in our Lorentzian metric signature the 
d'Alembertian
\begin{align}
\square^2 \coloneqq \rbr{\dd + \dd^{\dagger}}^2 =
	\dd \dd^{\dagger} + \dd^{\dagger} \dd\,.
\end{align}
\begin{align}
\dd{\star F} = \dd\star \dd A = \star (-)^2 \star^{-1}\dd\star\dd A =
	\star \dd^{\dagger} \dd A = \star\rbr{\square^2 - \dd \dd^{\dagger}} A\,.
\end{align}
One would like to have $\dd \dd^{\dagger} A = 0$, or $\dd^{\dagger} A = 
\text{const}$. This would be fulfilled if
\begin{align}
\dd^{\dagger} A = 0\,,
\end{align}
which is the Lorenz gauge\cite{Lorenz1867,Bladel1991a,Bladel1991b}.

%123456789012345678901234567890123456789
\subsection[$T(D)$ Noether current]{$\msansT(D)$ Noether current}

%123456789012345678901234567890123456789
The action is invariant up to a total differential
\begin{align}
\dva_{\lambda} S = - \int \lambda\,\dd\rbr{z \intprod \mfrakL}
= \int \lambda\,\frac{1}{2}\,\dd\cbr{
	+ z\intprod \rbr{F\wedge\star F} }\,,
\label{eq:tot-diff-em-10}
\end{align}
under the rigid infinitesimal transformation combined with a gauge 
transformation \cite[eq.\ 3.46]{Scheck2017}
\begin{align}
\dva A = - \lambda\cbr{z\intprod \dd \rbr{A-\dd\Lambda} + 
	\dd\sbr{z\intprod\rbr{A-\dd\Lambda}}}\,.
\end{align}
Choosing
\begin{align}
\dd\Lambda = \rfun{A}{z}\,z^{\flat}
\end{align}
makes the second term vanish, yielding
\begin{align}
\dva A = - \lambda\,z\intprod \dd A = - \lambda\,z\intprod F \,.
\end{align}

When the equations of motion are satisfied, infinitesimal 
transformation leads to
\begin{align}
\dva_{\lambda} S = \int \lambda\,\dd\cbr{+\rbr{z\intprod F}\wedge\star F}\,.
\end{align}
Taking the difference with \cref{eq:tot-diff-em-10} and setting $z = 
\mscre_{\alpha}$ yield
\begin{align}
0 &= \int \lambda\,\dd \mfrakT_{\alpha}, \\
\begin{split}
2 \mfrakT_{\alpha} &\coloneqq +\rbr{\mscre_{\alpha}\intprod F}\wedge\star F -
	F \wedge\rbr{\mscre_{\alpha}\intprod\star F}\,.
\end{split}
\end{align}
The continuity equation reads
\begin{align}
\dd \mfrakT_{\alpha} = 0\,.
\end{align}
In components, one needs
\begin{align}
\mscre_{\alpha}\intprod F &= F_{\alpha\beta}\,\vartheta^{\beta}\,,
\\
\mscre_{\alpha}\intprod \star F &=
	\frac{1}{2} F^{\beta\gamma}\,\epsilon_{\beta\gamma\alpha}\,;
\end{align}
equipped with \crefrange{eq:id-cof-10}{eq:id-cof-10}, one arrives at
\begin{align}
\rbr{\mscre_{\alpha}\intprod F}\wedge \star F &=
	- F_{\alpha\gamma} F^{\gamma\beta}\,\epsilon_{\beta}\,,
\\
F\wedge\rbr{\mscre_{\alpha}\intprod\star F} &=
	\frac{1}{2} F_{\beta\gamma} F^{\beta\gamma}\,\epsilon_{\alpha} +
	F_{\alpha\gamma} F^{\gamma\beta}\,\epsilon_{\beta}\,.
\end{align}
One finally has
\begin{align}
\mfrakT_{\alpha} =
	-F_{\alpha\gamma} F^{\gamma\beta}\,\epsilon_{\beta}
	-\frac{1}{4} F_{\beta\gamma} F^{\beta\gamma}\,\epsilon_{\alpha} \,.
\end{align}


%1234567890123456789012345678901234567890123456789012345678901234567890123456789
\section[$U(1)$-gauged complex scalar field theory]%
{$\msansU(1)$-gauged complex scalar field theory}

%1234567890123456789012345678901234567890123456789012345678901234567890123456789
The action reads
\begin{align}
S = \int -\DD\phi^*\wedge\star\DD\phi - m^2 \phi^*\wedge\star\phi
	-\frac{1}{2} F \wedge \star F\,,
\end{align}
where the exterior covariant derivative is defined in 
\cref{eq:def-ext-cov-der-bun-10}.

Variation does \emph{not} commute with exterior covariant derivative.
\begin{align}
\sbr{\dva, \DD}\phi = -\ii e \dva A\phi\,.
\end{align}

Variation of the covariant kinetic term reads
\begin{align}
\begin{split}
\dva \rbr{\DD \phi^*\wedge\star\DD\phi} &= 
\dd\rbr{\dva\phi^* \star\DD\phi + \dva\phi \star\DD\phi^*} 
\\ &\quad\,+
	\dva\phi^*\wedge\star\DD^{\dagger}\DD\phi +
	\dva\phi\wedge\star\DD^{\dagger}\DD\phi^* +
	\dva A \wedge \mfrakJ_A \,,
\end{split}
\end{align}
where the covariant codifferential is defined in \cref{eq:def-cov-cod-10}, and
\begin{align}
\mfrakJ_A &\coloneqq 
	\ii e \rbr{\phi^* \star\DD\phi - \phi\star\DD\phi^*}\,.
\end{align}

A generic variation of the action then reads
\begin{align}
\dva S &=
\int \dd\rbr{-\dva\phi^*\wedge\star\DD\phi -
\dva\phi\wedge\star\DD\phi^* - \dva A \wedge\star F}
\nonumber \\
&\quad\,-
\dva\phi^*\wedge\star\rbr{\DD^{\dagger}\DD + m^2}\phi -
\dva\phi\wedge\star\rbr{\DD^{\dagger}\DD + m^2}\phi^*
\nonumber \\
&\quad\,-
\dva A\wedge\rbr{\dd{\star F} - \mfrakJ_{A}}.
\end{align}


%123456789012345678901234567890123456789
\subsection{Noether's invariances}

%123456789012345678901234567890123456789

The action is invariant under the generic transformation
\begin{align}
\phi \to \ee^{-\ii e \Lambda} \phi\,,\quad
\phi^* \to \ee^{+\ii e \Lambda} \phi^*\,,\quad
A \to A - \dd \Lambda\,.
\end{align}

There are two scenarios \cite{Avery2015}.

If the transformation is rigid, $\dd \Lambda = 0$, one obtains $\mfrakJ_A$ as 
the Noether current from the boundary term as before, which satisfies the 
continuity equation $\dd \mfrakJ_A = 0$.

If the transformation is gauge with a compact support, the boundary term can be
dropped, and one obtains a Noether identity, which is under construction.

%\begin{align}
%\begin{split}
%\Sigma_{\alpha} &=
%	-\DD\phi^* \wedge \cbr{\star\rbr{\mscre_{\alpha}\intprod \DD\phi} +
%		\rbr{\mscre_{\alpha}\intprod\star\DD\phi}}
%-m^2\phi^*\phi\,\epsilon_{\alpha}
%\\
%&\quad\,
%- \frac{1}{2} F \wedge \cbr{\star\rbr{\mscre_{\alpha}\intprod F} -
%		\rbr{\mscre_{\alpha}\intprod \star F}}\,.
%\end{split}
%\end{align}

%123456789012345678901234567890123456789
%\subsection[$T(D)$ Noether current]{$\msansT(D)$ Noether current}

%123456789012345678901234567890123456789

\cite{Avery2015}


\appendix

%1234567890123456789012345678901234567890123456789012345678901234567890123456789
\section{Differential geometry}

%1234567890123456789012345678901234567890123456789012345678901234567890123456789

\subsection{Basic operations}

Following the style of \cite{Burke1985}, it is attempted to define 
the operations \emph{inductively}, instead of giving explicit expressions, 
which is much less effective than the former in practice.

\paragraph{Interior product}

Let $z$ be a vector, $\omega$ be a $1$-form, $\chi$ be a $k$-form. 
The \emph{interior product} is defined inductively as the bilinear map
satisfying
\begin{align}
z \intprod 1 &= 0\,,
\\
z \intprod \omega &\coloneqq \rfun{\omega}{z}\,,
\\
z \intprod \rbr{\omega \wedge \chi} &\coloneqq
	\rbr{z\intprod\omega}\wedge \chi - \omega \wedge \rbr{z\intprod \chi}\,.
\label{eq:anti-prod-100}
\end{align}
\Cref{eq:anti-prod-100} is also known as the anti-product rule.

By induction one can show that for a $p$-form $\phi$,
\begin{align}
z \intprod \rbr{\phi \wedge \chi} =
	\rbr{z\intprod\phi}\wedge \chi +
		\rbr{-}^{p} \chi \wedge \rbr{z\intprod \chi}\,.
\end{align}

\paragraph{Hodge star}

Let $\omega$ be an $1$-form, $\chi$ be a $k$-form. The Hodge star
$\star$ is defined inductively as the linear map 
\begin{align}
\star 1 &\coloneqq \mathrm{vol}\,, \\
\star\rbr{\chi\wedge\omega} &\coloneqq \omega^{\sharp} \intprod \star\chi\,.
\label{eq:hodge-100}
\end{align}

Non-gravitational theories features $\sbr{\dva, \star} = 0$, which means
\cite[sec.\ 3.2]{Muench1998}
\begin{align}
\dva g_{\mu\nu} = -2\omega_{(\mu\nu)},\qquad 
\dva\vartheta^{\mu} = \omega_{\nu}{}^{\mu}\,\vartheta^{\nu}\,;
\end{align}
for an orthonormal coframe, the allowed variations are 
$\omega_{(\alpha\beta)} = 0$.


\paragraph{Identity}

Inspired by \cite[eq.\ (3.167)]{Straumann2013}, for a $1$-form $\omega$,
$k$-form $\chi$, one can derive
\begin{align}
\omega\wedge\star\chi &= \rbr{-}^{D-k}\star\chi\wedge\omega
= \rbr{-}^{D-k} \star^{-1}
	\rbr{\omega^\sharp\intprod\star\star\chi}
\nonumber \\
&= \rbr{-}^{D-k} \rbr{-}^{(D-k-1)(k+1)+s}\star
	\rbr{\omega^\sharp\intprod \rbr{-}^{k(D-k)+s}
		\star^{-1}\star\chi }
\nonumber \\
&= \rbr{-}^{k+1} \star\rbr{\omega^\sharp\intprod \chi }\,.
\label{eq:ident-10}
\end{align}
This will be useful in \crefrange{eq:id-cof-10}{eq:id-cof-30}.


\paragraph{Codifferential}
The \emph{codifferential} $\dd^\dagger$ is defined as follows. Let $\phi$ be a 
$k$-form, $\chi$ be a $(k-1)$-form.
\begin{align}
\dd\rbr{\chi^* \wedge \star \phi} &\equiv
\dd \chi^* \wedge \star \phi - (-)^{k} \chi^* \wedge \dd\star\phi
\eqqcolon
\dd \chi^* \wedge \star \phi - \chi^* \wedge \star\dd^{\dagger}\phi
\\
&=
\dd \chi^* \wedge \star \phi - \chi^* \wedge 
	\star (-)^{k} \star^{-1} \dd \star \phi\,.
\end{align}
In other words,
\begin{empheq}[box=\fbox]{align}
\dd^{\dagger} \phi = (-)^{k} \star^{-1} \dd \star \phi\,.
\label{eq:codiff-def-10}
\end{empheq}
If $\int \dd\rbr{\chi^* \wedge \star \phi} = 0$, then $\dd^\dagger$ is the 
adjoint of the exterior derivative $\dd$ in functional sense.

%123456789012345678901234567890123456789
\subsection{Coframe bases}

%123456789012345678901234567890123456789

Coframes come into play if an explicit expression in components is wanted, or
if gravitation comes into play.

For a detailed discussion about the meaning of untwisted (describing 
extensions) and twisted (describing densities) differential forms, see e.g.\
\cite[sec.\ 22, 28]{Burke1985}.

\paragraph{Untwisted orthonormal $k$-cobases}

Let $\cbr{\vartheta^{\alpha}}$ be an orthonormal coframe. The orthonormal 
basis for untwisted $k$-form is defined inductively as
\begin{align}
1&\,, \\
\vartheta^{\alpha_{1}\alpha_{2}\ldots \alpha_{k}} &\coloneqq 
	\vartheta^{\alpha_{1}} \wedge \vartheta^{\alpha_{2}\ldots \alpha_{k}}\,.
\end{align}

\paragraph{Twisted orthonormal $k$-cobases}
Let $\cbr{\vartheta^{\alpha}}$ be an orthonormal coframe. The orthonormal 
basis for twisted $\rbr{D-k}$-form is defined inductively as
\begin{align}
\epsilon &\coloneqq \mathrm{vol}\,, \\
\epsilon_{\alpha_{1}\alpha_{2}\ldots \alpha_{k}} &\coloneqq
\mscre_{\alpha_{k}} \intprod \epsilon_{\alpha_{1}\ldots \alpha_{k-1}}\,.
\end{align}

By using \cref{eq:hodge-100} and induction, one can show that
\begin{align}
\epsilon_{\alpha_{1}\alpha_{2}\ldots \alpha_{k}} =
\star \vartheta_{\alpha_{1}\alpha_{2}\ldots \alpha_{k}}\,.
\end{align}

\paragraph{Identities}
The following identities come from \cref{eq:ident-10}
\begin{align}
\vartheta^{\alpha}\wedge\epsilon_{\beta\gamma}
&= -\delta^{\alpha}_{\beta}\,\epsilon_{\gamma}
	+\delta^{\alpha}_{\gamma}\,\epsilon_{\beta}\,,
\label{eq:id-cof-10}
\\
\vartheta^{\alpha}\wedge\epsilon_{\beta\gamma\delta} &=
\delta^{\alpha}_{\beta}\,\epsilon_{\gamma\delta} -
\delta^{\alpha}_{\gamma}\,\epsilon_{\beta\delta} +
\delta^{\alpha}_{\delta}\,\epsilon_{\beta\gamma}\,.
\label{eq:id-cof-20}
\end{align}
One can further deduce that
\begin{align}
\begin{split}
\vartheta^{\alpha\beta}\wedge\epsilon_{\gamma\delta\epsilon} &= +
\rbr{\delta^{\alpha}_{\gamma}\delta^{\beta}_{\delta}-
	\delta^{\alpha}_{\delta}\delta^{\beta}_{\gamma}}\epsilon_{\epsilon}
\\
&\quad\, -
\rbr{\delta^{\alpha}_{\gamma}\delta^{\beta}_{\epsilon}-
	\delta^{\alpha}_{\epsilon}\delta^{\beta}_{\gamma}}\epsilon_{\delta}
\\
&\quad\, +
\rbr{\delta^{\alpha}_{\delta}\delta^{\beta}_{\epsilon}-
	\delta^{\alpha}_{\epsilon}\delta^{\beta}_{\delta}}\epsilon_{\gamma}\,.
\end{split}
\label{eq:id-cof-30}
\end{align}
\Crefrange{eq:id-cof-10}{eq:id-cof-30} are useful in constructing explicit 
expressions in components.

\paragraph{Translation}

Consider the rigid infinitesimal transformation
\begin{align}
\dva_{\lambda}\chi = \lambda\, \pounds_z \chi =
	\lambda\sbr{z\intprod\dd\chi + \dd\rbr{z\intprod\chi}}\,;
\end{align}
`rigid' means
\begin{align}
\dva_{\lambda} \vartheta^{\alpha} = 0\,.
\end{align}


\subsection{Connection on the principal bundle}

\paragraph{Exterior covariant derivative}
Let $\chi$ be a $\BbbC$-valued $k$-form. The 
\emph{exterior covariant derivative} of 
$\chi$ reads
\begin{align}
\DD\chi \coloneqq \rbr{\dd - \ii e A}\chi,
\qquad
\DD\chi^* \coloneqq \rbr{\dd + \ii e A}\chi^*\,,
\label{eq:def-ext-cov-der-bun-10}
\end{align}
where $A$ is a $\mfraku(1)$-valued \emph{connection form}.

\paragraph{Covariant codifferential}
The \emph{covariant codifferential} $\DD^\dagger$ is defined as follows. 
Let $\phi$ be a $\BbbC$-valued $k$-form, $\chi$ be a 
$\BbbC$-valued $(k-1)$-form.
\begin{align}
\dd\rbr{\chi^* \wedge \star \phi} &\equiv
\dd \chi^* \wedge \star \phi - (-)^{k} \chi^* \wedge \dd\star\phi
\eqqcolon
\DD \chi^* \wedge \star \phi - \chi^* \wedge \star\DD^{\dagger}\phi
\\
&=
\DD \chi^* \wedge \star \phi - \ii e A \wedge \chi^* \wedge \star \phi -
	(-)^{k} \chi^* \wedge \dd\star\phi
\nonumber \\
&=
\DD \chi^* \wedge \star \phi +
	\chi^* \wedge (-)^{k} \ii e A \star \phi -
	(-)^{k} \chi^* \wedge \dd\star\phi
\nonumber \\
&=
\DD \chi^* \wedge \star \phi - \chi^* \wedge 
	\star (-)^{k} \star^{-1} \rbr{\dd - \ii e A} \star \phi\,.
\end{align}
In other words,
\begin{empheq}[box=\fbox]{align}
\DD^{\dagger} \phi = (-)^{k} \star^{-1}
	\rbr{\dd - \ii e A} \star \phi\,.
\label{eq:def-cov-cod-10}
\end{empheq}
If $\int \dd\rbr{\chi^* \wedge \star \phi} = 0$, $\DD^\dagger$ becomes the 
adjoint of the exterior covariant derivative $\DD$ in the functional sense.


%123456789012345678901234567890123456789
\subsection{Variation of coframe bases}

%123456789012345678901234567890123456789

Upon variation of $\vartheta^{\alpha}$, $\vartheta^{\alpha_{1}\alpha_{2}\ldots 
\alpha_{k}}$ goes under
\begin{align}
\dva \vartheta^{\alpha_{1}\alpha_{2}\ldots \alpha_{k}} &=
\dva \vartheta^{\alpha} \wedge \rbr{\mscre_{\alpha}\intprod 
	\vartheta^{\alpha_{1}\alpha_{2}\ldots \alpha_{k}}}\,,
\end{align}
which can be proved by induction.

Upon variation of $\vartheta^{\alpha}$, $\epsilon_{\alpha_{1}\alpha_{2}\ldots 
\alpha_{k}}$ goes under \cite[sec.\ A.2]{Muench1998}
\begin{align}
\dva \epsilon_{\alpha_{1}\alpha_{2}\ldots \alpha_{k}} &=
\dva \vartheta^{\alpha} \wedge \rbr{\mscre_{\alpha}\intprod 
	\epsilon_{\alpha_{1}\alpha_{2}\ldots \alpha_{k}}}\,.
\end{align}

\paragraph{Variation of Hodge star}
In gravitational theories \cite[sec.\ 3.2]{Muench1998} with an orthonormal 
cobasis,
\begin{align}
\sbr{\dva, \star}\phi = 
\dva\vartheta^{\alpha}\wedge\rbr{\mscre_{\alpha}\intprod\star\phi} -
\star\rbr{\dva\vartheta^{\alpha}\wedge\rbr{\mscre_{\alpha}\intprod\phi}}\,.
\end{align}

Let $\chi$ be a $p$-form, $\phi$ another form \cite[sec.\ 5]{Itin1999}.
\begin{align}\
\dva\rbr{\chi\wedge\star\phi} &=
\dva \chi \wedge \star \phi + \dva \phi \wedge \star \chi -
	\dva \vartheta^{\alpha} \wedge \Sigma_{\alpha}\,,
\\
\Sigma_{\alpha} &\coloneqq \chi \wedge \cbr{
	\star\rbr{\mscre_{\alpha}\intprod\phi} -
	\rbr{-}^{p} \rbr{\mscre_{\alpha}\intprod\star\phi}}\,.
\end{align}


% Let's print the overall heading of the bibliography first:
%\printbibheading
\printbibliography

\end{document}