\documentclass[a4paper]{article}

\input{./preambles/preamble}
\usetikzlibrary{external,
angles,
decorations.pathmorphing,
calc,intersections,through,backgrounds,
decorations.pathreplacing,
patterns,shadings,
arrows,
shapes.geometric}
% save compiled tikz plots; enable --shell-escape
\tikzexternalize[prefix=./tikz/]
% Sebastian did it
\usepackage[outline]{contour}
\contourlength{2pt}

\input{./preambles/unicode}
	\setmathfont{XITS Math}[range={
		"02A30-"02A3E}] % ⨼ etc.

\setdefaultlanguage{english}
\setotherlanguages{german,french,italian,greek,latin,russian}
\setmainfont{CMU Serif}
\setsansfont{CMU Sans Serif}
% \newfontfamily{\greekfont}{CMU Serif}
% \newfontfamily{\greekfontsf}{CMU Sans Serif}

\input{./preambles/math-single}
\input{./preambles/math-brac}
\input{./preambles/math-thm}
\input{./preambles/phys-chem}

\newcommand{\RomaN}[1]{%
  \textup{\uppercase\expandafter{\romannumeral#1}}%
}

\usepackage[
	bibencoding  = auto,
	backend      = biber,
	sorting      = nyt,
	sortlocale   = auto,
	hyperref     = true,
	backref      = true,
	style        = phys,
	doi          = false,
	eprint       = true,
	isbn         = true,
	url          = true,
	subentry     = true,
	articletitle = true,
	biblabel     = brackets,
	chaptertitle = true,
	pageranges   = true,
	%refsegment   = section,
]{biblatex}
\addbibresource{main.bib}

%\input{./preambles/biblatex-user}

% \setcounter{secnumdepth}{6}
% So that paragraph is also numbered

%\numberwithin{equation}{subsection}

\title{Differential form}
\author{Yi-Fan Wang}


\begin{document}
\maketitle

%\begin{abstract}
%Your abstract.
%\end{abstract}

\tableofcontents

%1234567890123456789012345678901234567890123456789012345678901234567890123456789
\section{Non-gravitational theories}

%1234567890123456789012345678901234567890123456789012345678901234567890123456789

%1234567890123456789012345678901234567890123456789012345678901234567890123456789
\subsection{Differential geometry}

%1234567890123456789012345678901234567890123456789012345678901234567890123456789

\paragraph{Interior product}

Let $z$ be a vector, $\omega$ be a $1$-form, $\chi$ be a $k$-form. 
The \emph{interior product} is defined inductively as the bilinear map
satisfying
\begin{align}
z \intprod \omega &\coloneqq \rfun{\omega}{z}\,, \\
z \intprod \rbr{\omega \wedge \chi} &\coloneqq
	\rbr{z\intprod\omega}\wedge \chi - \omega \wedge \rbr{z\intprod \chi}\,.
\label{eq:anti-prod-100}
\end{align}
\Cref{eq:anti-prod-100} is also known as the anti-product rule.

By inductiion one can show that for a $p$-form $\phi$,
\begin{align}
z \intprod \rbr{\phi \wedge \chi} =
	\rbr{z\intprod\phi}\wedge \chi +
		\rbr{-}^{p} \chi \wedge \rbr{z\intprod \chi}\,.
\end{align}

\paragraph{Hodge star}

Let $\omega$ be an $1$-form, $\chi$ be a $k$-form. The Hodge star
$\star$ is defined inductively as the linear map \cite[sec.\ 24]{Burke1985}
\begin{align}
\star 1 &\coloneqq \mathrm{vol}\,, \\
\star\rbr{\chi\wedge\omega} &\coloneqq \omega^{\sharp} \intprod \star\chi\,.
\label{eq:hodge-100}
\end{align}

Non-gravitational theories features $\sbr{\dva, \star} = 0$, which means
\cite[sec.\ 3.2]{Muench1998}
\begin{align}
\dva g_{\mu\nu} = -2\omega_{(\mu\nu)},\qquad 
\dva\vartheta^{\mu} = \omega_{\nu}{}^{\mu}\,\vartheta^{\nu}\,;
\end{align}
for an orthonormal coframe, the allowed variations are 
$\omega_{(\alpha\beta)} = 0$.

\subsection{Connection on the principal bundle}

\paragraph{Covariant differential}
Let $\chi$ be a $\BbbC$-valued $k$-form. The \emph{covariant differential} of 
$\chi$ reads
\begin{align}
\DD\chi \coloneqq \rbr{\dd - \ii e A}\wedge \chi,
\qquad
\DD\chi^* \coloneqq \rbr{\dd + \ii e A}\wedge \chi^*\,,
\end{align}
where $A$ is a $\mfraku(1)$-valued \emph{connection form}.

\paragraph{Covariant codifferential}
The \emph{covariant codifferential} $\DD^\dagger$ is the adjoint of the 
covariant differential $\DD$ in the following sense. Let $\phi$ be a 
$\BbbC$-valued $k$-form, $\chi$ be a $\BbbC$-valued $(k-1)$-form.
\begin{align}
\int \dd\rbr{\chi^* \wedge \star \phi} &\equiv
\int \dd \chi^* \wedge \star \phi - (-)^{k} \chi^* \wedge \dd\star\phi
\eqqcolon
\int \DD \chi^* \wedge \star \phi - \chi^* \wedge \star\DD^{\dagger}\phi
\\
&=
\int \DD \chi^* \wedge \star \phi - \ii e A \wedge \chi^* \wedge \star \phi -
	(-)^{k} \chi^* \wedge \dd\star\phi
\nonumber \\
&=
\int \DD \chi^* \wedge \star \phi +
	\chi^* \wedge (-)^{k} \ii e A \wedge \star \phi -
	(-)^{k} \chi^* \wedge \dd\star\phi
\nonumber \\
&=
\int \DD \chi^* \wedge \star \phi - \chi^* \wedge 
	\star (-)^{k} \star^{-1} \rbr{\dd - \ii e A} \wedge \star \phi\,.
\end{align}
\begin{empheq}[box=\fbox]{align}
\DD^{\dagger} \phi = (-)^{k} \star^{-1}
	\rbr{\dd - \ii e A} \wedge \star \phi\,.
\end{empheq}

\subsection{Maxwell--Klein--Fock--Gordon theory}
\begin{align}
S = \int -\DD\phi^*\wedge\star\DD\phi - m^2 \phi^*\wedge\star\phi
	-\frac{1}{2} F \wedge \star F.
\end{align}

\begin{align}
\dva \DD\phi = -\ii e \dva A\phi + \DD \dva \phi.
\end{align}

\begin{align}
\begin{split}
\dva \rbr{\DD \phi^*\wedge\star\DD\phi} &= 
\dd\rbr{\dva\phi^*\wedge\star\DD\phi + \dva\phi\wedge\star\DD\phi^*}
\\
&\quad\,-
\dva\phi^*\wedge\star\DD^{\dagger}\DD\phi -
\dva\phi\wedge\star\DD^{\dagger}\DD\phi^*
\\
&\quad\,+
\dva A \wedge\rbr{\ii e \rbr{\phi^* \star\DD\phi - \phi\star\DD\phi^*}}\,,
\end{split}
\\
\dva \rbr{F\wedge\star F} &=
	2 \dd\rbr{\dva A \wedge \star F} - 2 \dva A \wedge \dd {\star F}\,.
\end{align}

\begin{align}
\dva S &=
\int -\dd\rbr{\dva\phi^*\wedge\star\DD\phi +
\dva\phi\wedge\star\DD\phi^* + \dva A \wedge\star F}
\nonumber \\
&\quad\,+
\dva\phi^*\wedge\star\rbr{\DD^{\dagger}\DD-m^2}\phi +
\dva\phi\wedge\star\rbr{\DD^{\dagger}\DD-m^2}\phi^*
\nonumber \\
&\quad\,+
\dva A\wedge\rbr{\dd{\star F} -
	\ii e \rbr{\phi^* \star\DD\phi - \phi \star\DD\phi^*}}.
\end{align}

\subsubsection{Lorenz gauge}
The Laplace--de Rham operator, or in our metric signature the d'Alembertian
\begin{align}
\square^2 \coloneqq \rbr{\dd + \dd^{\dagger}}^2 =
	\dd \dd^{\dagger} + \dd^{\dagger} \dd\,.
\end{align}
\begin{align}
\dd{\star F} = \dd\star \dd A = \star (-)^2 \star^{-1}\dd\star\dd A =
	\star \dd^{\dagger} \dd A = \star\rbr{\square^2 - \dd \dd^{\dagger}} A\,.
\end{align}
One would like to have $\dd \dd^{\dagger} A = 0$, or $\dd^{\dagger} A = 
\text{const}$. This would be fulfiled if
\begin{align}
\dd^{\dagger} A = 0\,,
\end{align}
which is the Lorenz gauge\cite{Lorenz1867,Bladel1991a,Bladel1991b}.

\subsubsection{Noether's first theorem}

\subsubsection{Noether's second theorem}


%1234567890123456789012345678901234567890123456789012345678901234567890123456789
\section{Gravitational theory}

%1234567890123456789012345678901234567890123456789012345678901234567890123456789

%1234567890123456789012345678901234567890123456789012345678901234567890123456789
\subsection{Differential forms}

%1234567890123456789012345678901234567890123456789012345678901234567890123456789

\paragraph{Untwisted orthonormal $k$-cobases}
Let $\cbr{\vartheta^{\alpha}}$ be an orthonormal coframe. The orthonormal 
basis for untwisted $k$-form is defined inductively as
\begin{align}
1&\,, \\
\vartheta^{\alpha_{1}\alpha_{2}\ldots \alpha_{k}} &\coloneqq 
	\vartheta^{\alpha_{1}} \wedge \vartheta^{\alpha_{2}\ldots \alpha_{k}}\,.
\end{align}

Upon variation of $\vartheta^{\alpha}$, $\vartheta^{\alpha_{1}\alpha_{2}\ldots 
\alpha_{k}}$ goes under
\begin{align}
\dva \vartheta^{\alpha_{1}\alpha_{2}\ldots \alpha_{k}} &=
\dva \vartheta^{\alpha} \wedge \rbr{\mscre_{\alpha}\intprod 
	\vartheta^{\alpha_{1}\alpha_{2}\ldots \alpha_{k}}}\,,
\end{align}
which can be proved by induction.

\paragraph{Twisted orthonormal $k$-cobases}
Let $\cbr{\vartheta^{\alpha}}$ be an orthonormal coframe. The orthonormal 
basis for twisted $\rbr{D-k}$-form is defined inductively as
\begin{align}
\epsilon &\coloneqq \mathrm{vol}\,, \\
\epsilon_{\alpha_{1}\alpha_{2}\ldots \alpha_{k}} &\coloneqq
\mscre_{\alpha_{k}} \intprod \epsilon_{\alpha_{1}\ldots \alpha_{k-1}}\,.
\end{align}

By using \cref{eq:hodge-100} and induction, one can show that
\begin{align}
\epsilon_{\alpha_{1}\alpha_{2}\ldots \alpha_{k}} =
\star \vartheta_{\alpha_{1}\alpha_{2}\ldots \alpha_{k}}\,.
\end{align}

Upon variation of $\vartheta^{\alpha}$, $\epsilon_{\alpha_{1}\alpha_{2}\ldots 
\alpha_{k}}$ goes under \cite[sec.\ A.2]{Muench1998}
\begin{align}
\dva \epsilon_{\alpha_{1}\alpha_{2}\ldots \alpha_{k}} &=
\dva \vartheta^{\alpha} \wedge \rbr{\mscre_{\alpha}\intprod 
	\epsilon_{\alpha_{1}\alpha_{2}\ldots \alpha_{k}}}\,.
\end{align}

\paragraph{Variation of Hodge star}
In gravitational theories \cite[sec.\ 3.2]{Muench1998} with an orthonormal 
cobasis,
\begin{align}
\sbr{\dva, \star}\phi = 
\dva\vartheta^{\alpha}\wedge\rbr{\mscre_{\alpha}\intprod\star\phi} -
\star\rbr{\dva\vartheta^{\alpha}\wedge\rbr{\mscre_{\alpha}\intprod\phi}}\,.
\end{align}

Let $\chi$ be a $p$-form, $\phi$ another form \cite[sec.\ 5]{Itin1999}.
\begin{align}\
\dva\rbr{\chi\wedge\star\phi} &=
\dva \chi \wedge \star \phi + \dva \phi \wedge \star \chi -
	\dva \vartheta^{\alpha} \wedge \Sigma_{\alpha}\,,
\\
\Sigma_{\alpha} &\coloneqq \chi \wedge \cbr{
	\star\rbr{\mscre_{\alpha}\intprod\phi} -
	\rbr{-}^{p} \rbr{\mscre_{\alpha}\intprod\star\phi}}\,.
\end{align}

%1234567890123456789012345678901234567890123456789012345678901234567890123456789
\subsection{Maxwell--Klein--Fock--Gordon theory}

%1234567890123456789012345678901234567890123456789012345678901234567890123456789

\begin{align}
\begin{split}
\Sigma_{\alpha} &=
	-\DD\phi^* \wedge \cbr{\star\rbr{\mscre_{\alpha}\intprod \DD\phi} +
		\rbr{\mscre_{\alpha}\intprod\star\DD\phi}}
-m^2\phi^*\phi\,\epsilon_{\alpha}
\\
&\quad\,
- \frac{1}{2} F \wedge \cbr{\star\rbr{\mscre_{\alpha}\intprod F} -
		\rbr{\mscre_{\alpha}\intprod \star F}}\,.
\end{split}
\end{align}

\subsubsection{Noether's first theorem}

\subsubsection{Noether's second theorem}

%\input{sections/intro}
%\printbibliography[segment=1,heading=subbibintoc]


% Let's print the overall heading of the bibliography first:
%\printbibheading
\printbibliography

\end{document}